\documentclass[12pt,a4paper]{article}

% --- Pakiety ---
\usepackage[polish]{babel}
\usepackage[T1]{fontenc}
\usepackage[utf8]{inputenc}
\usepackage{geometry}
\usepackage{hyperref}
\usepackage{graphicx}
\usepackage{listings}
\usepackage{xcolor}

% --- Ustawienia strony ---
\geometry{margin=2.5cm}

% --- Ustawienia listings ---
\lstset{
  basicstyle=\ttfamily\small,
  backgroundcolor=\color{gray!10},
  frame=single,
  breaklines=true
}

\title{Instrukcja instalacji i konfiguracji maszyny wirtualnej\\
System \textbf{Ubuntu} w \textbf{VirtualBox}}
\author{Autor: \rule{6cm}{0.4pt}}
\date{\today}

\begin{document}
\maketitle
\tableofcontents
\newpage

\section{Wprowadzenie}
Celem niniejszej instrukcji jest przedstawienie krok po kroku procesu instalacji oraz podstawowej konfiguracji maszyny wirtualnej z systemem \textbf{Ubuntu} przy użyciu oprogramowania \textbf{Oracle VM VirtualBox}.

\section{Wymagania systemowe}
\begin{itemize}
  \item Komputer z systemem Windows / Linux / macOS
  \item Minimum 8 GB wolnego miejsca na dysku
  \item Minimum 4 GB pamięci RAM (zalecane 8 GB)
  \item Dostęp do Internetu
\end{itemize}

\section{Pobieranie wymaganych plików}
\subsection{VirtualBox}
\begin{enumerate}
  \item Wejdź na stronę: \url{https://www.virtualbox.org}
  \item Pobierz najnowszą wersję VirtualBox dla swojego systemu operacyjnego
  \item Zainstaluj program, postępując zgodnie z instrukcjami instalatora
\end{enumerate}

\subsection{Obraz ISO Ubuntu}
\begin{enumerate}
  \item Wejdź na stronę: \url{https://ubuntu.com/download}
  \item Pobierz obraz ISO wybranej wersji Ubuntu (np. Ubuntu 22.04 LTS)
\end{enumerate}

\section{Tworzenie nowej maszyny wirtualnej}
\begin{enumerate}
  \item Uruchom VirtualBox
  \item Kliknij \textbf{New / Nowa}
  \item Nazwa maszyny: \textit{Ubuntu}
  \item Typ: \textit{Linux}
  \item Wersja: \textit{Ubuntu (64-bit)}
\end{enumerate}

\subsection{Konfiguracja pamięci RAM i CPU}
\begin{itemize}
  \item Pamięć RAM: minimum 2048 MB (zalecane 4096 MB)
  \item Procesory: 2 lub więcej (jeśli dostępne)
\end{itemize}

\section{Konfiguracja dysku wirtualnego}
\begin{enumerate}
  \item Wybierz \textbf{Utwórz nowy dysk wirtualny}
  \item Typ pliku: \textbf{VDI}
  \item Alokacja: \textbf{Dynamicznie przydzielany}
  \item Rozmiar: minimum 20 GB
\end{enumerate}

\section{Podłączanie obrazu ISO}
\begin{enumerate}
  \item Wejdź w \textbf{Ustawienia} maszyny wirtualnej
  \item Zakładka \textbf{Nośniki}
  \item Dodaj pobrany plik ISO Ubuntu jako napęd optyczny
\end{enumerate}

\section{Instalacja systemu Ubuntu}
\begin{enumerate}
  \item Uruchom maszynę wirtualną
  \item Wybierz opcję \textbf{Install Ubuntu}
  \item Wybierz język, układ klawiatury oraz połączenie sieciowe
  \item Wybierz \textbf{Normal installation}
  \item Typ instalacji: \textbf{Erase disk and install Ubuntu}
  \item Utwórz konto użytkownika
  \item Rozpocznij instalację i poczekaj na jej zakończenie
\end{enumerate}

\section{Konfiguracja po instalacji}
\subsection{Instalacja dodatków gościa}
\begin{enumerate}
  \item W menu VirtualBox wybierz \textbf{Devices $\rightarrow$ Insert Guest Additions CD Image}
  \item Uruchom instalator dodatków
  \item Zrestartuj maszynę wirtualną
\end{enumerate}

\subsection{Ustawienia ekranu i schowka}
\begin{itemize}
  \item Ustaw tryb pełnoekranowy
  \item Włącz współdzielony schowek (Clipboard: Bidirectional)
\end{itemize}

\section{Podsumowanie}
Po wykonaniu powyższych kroków maszyna wirtualna z systemem Ubuntu jest gotowa do użytku. Można na niej instalować dodatkowe oprogramowanie oraz wykorzystywać ją do nauki i testów.

\end{document}
